% Cuento de Navidad en LaTeX
% Guarda este archivo como cuento_navidad.tex y compílalo con pdflatex o xelatex.
\documentclass[12pt]{article}
\usepackage[spanish]{babel}
% Permite acentos en español y buscar
\usepackage[utf8]{inputenc}
\usepackage[T1]{fontenc}
% \usepackage{lmodern}
% \usepackage{microtype}
\usepackage{geometry}
\geometry{margin=1in}
\usepackage{setspace}
\setstretch{1.25}
\usepackage{parskip}
\usepackage{tcolorbox}
\begin{document}
\title{La Estrella que aprendió a escuchar}
\author{Un Cuento de Navidad}
\date{}
\maketitle

\section*{Víspera en el Valle de las Nubes}
Era la víspera de Navidad en el Valle de las Nubes. La nieve cubría las calles y los techos, y el aire olía a canela y hogaza recién horneada. En una casa de madera, un niño llamado Iker miraba por la ventana sin poder dormirse. Quería creer en las historias de su abuela, esas historias en las que la noche se ilumina con una promesa.

\section*{La estrella caída}
Al filo de la medianoche, un destello rozó el jardín: una estrella, pequeña como una bellota, tibia como una vela. Iker la encontró junto a la puerta, brillando con un color ámbar. La estrella habló con voz suave, como campanillas: \emph{— No temas, niño. Soy una estrella caída. Quiero volver al cielo, pero para ello necesito que la gente aprenda a escuchar y a compartir.}
\emph{—Escuchar y compartir? ¿Cómo puedo ayudar a una estrella?}
\emph{— Con cada gesto amable que hagas tú, yo me haré más fuerte. Empecemos ahora.}

\begin{tcolorbox}[colback=yellow!5!white,colframe=orange!75!black,title=La primera tarea de Iker]
Iker dejó la estrella sobre la mesa de la sala y salió de inmediato. Su primera parada fue la casa de Doña Rosa, la vecina mayor que vivía al final de la calle. Le llevó unas tortas de chocolate y una manta; Doña Rosa sonrió con lágrimas de gratitud y, por primera vez en semanas, se sintió acompañada.
\end{tcolorbox}

\section*{La ruta de la bondad}
La estrella dejó un rayo cálido que marcaba el camino de Iker, y él siguió iluminado por su luz. Visitó a Don Sergio, el panadero, que había cerrado la puerta de la panadería por cansancio. Le entregó un cuenco de chocolate caliente y escuchó con paciencia sus historias de la juventud. Después fue a la escuela, donde la maestra Señora Lara estaba cansada de tantas tareas; con una frase de ánimo, la estrella brilló un poco más y el aula pareció respirar de nuevo.

\begin{quote}
\emph{La Navidad no es el oro de las vitrinas, sino el oro de los gestos que compartimos con otros.}
\end{quote}

Con cada visita, la estrella ganaba fuerza, y su luz se volvía una ruta de estrellas dibujada en el cielo de la plaza.

\section*{La llegada al mirador de la iglesia}
Aquella noche, la estrella guiaba a Iker hacia la iglesia del pueblo. Cada ventana tenía una historia: niños que reían, ancianos que recordaban, parejas que se tomaban de la mano. En el campanario, la estrella dejó caer una lluvia de luz que unió las casas como si cada hogar sostuviera una cuerda invisible de amistad.
\begin{tcolorbox}[colback=blue!5!white,colframe=blue!75!black,title=El mensaje de la estrella]
\emph{— Hemos aprendido que lo más valioso de la Navidad no son los regalos, sino escuchar a los demás y compartir lo que tenemos. Ese es el modo de mantener la luz encendida.}
\end{tcolorbox}

\section*{El regreso de la estrella}
Ya cerca del amanecer, la estrella dijo: \emph{— Es hora de volver al cielo. Pero no te olvides de lo que aprendiste: cada acto de escucha y cada gesto de cuidado mantiene viva mi luz.}
Iker volvió a casa con el corazón caliente y una nueva costumbre: cada día, antes de dormir, pensar en una persona a la que podría ayudar y en un pequeño gesto para hacerla sonreír.

\section*{Epílogo}
La Navidad llegó de verdad a quien la buscó en el interior de su propia casa. Los vecinos, cuando se miraban a los ojos, entendieron que la estrella no era una promesa lejana, sino una voz interior que les recordaba que la bondad bien guardada merece ser compartida. Y cuando el sol asomó sobre las montañas, la estrella, ya en su lugar en el firmamento, brillaba apenas, como un recuerdo que se repite cada año: la magia de la Navidad está en escuchar y en dar.

\end{document}